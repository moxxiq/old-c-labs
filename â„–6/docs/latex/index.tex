\begin{center} \subsection*{Варіант №4}\end{center} 

\begin{center} \end{center}  Виконав студент групи КМ-\/82\+: {\bfseries Бубела Дмитро}~\newline
 \subsubsection*{Завдання\+:}

\subsubsection*{Дано матрицю A розмірності m$\ast$n. Обчислити суми елементів кожного стовпця.}

\subsubsection*{Визначити найбільше значення цих сум і номер відповідного стовпця.}

\paragraph*{Виконується функцією \hyperlink{main_8c_a2e10594dc040249a898e2880b4c64322}{task\+\_\+1()}}

Для повтору виконання використовується функція \hyperlink{lab__functions_8h_a770281b98587f9f65ca4cc75b1d061db}{to\+\_\+repeat()}~\newline
Користувач вводить розмірність майбутнього двумірного масива. Він створюється та заразом масив сум стовбчиків, після чого зануляється. Виконується заповнення випадковими числами та підрахунок сум стовбців.~\newline
Визначається колонка з максимальною сумою елементів.~\newline
Виконується вивід масиву з виділенням стовбця з максимальною сумою.~\newline
Виконується вивід масиву сум, вивід стовбця з найбільшою сумою елементів~\newline
Виконується вивільнення пам\textquotesingle{}яті.~\newline
\paragraph*{Нижче наведений сирцевий код прорами}


\begin{DoxyCodeInclude}
1 \textcolor{preprocessor}{#include <stdio.h>}
2 \textcolor{preprocessor}{#include <stdlib.h>}
3 \textcolor{preprocessor}{#include <time.h>}
4 \textcolor{preprocessor}{#include "\hyperlink{lab__functions_8h}{lab\_functions.h}"}
6 
17 \textcolor{keywordtype}{void} \hyperlink{main_8c_a2e10594dc040249a898e2880b4c64322}{task\_1}()\{
18     \textcolor{keywordtype}{int} cols, rows, max\_col = 0;
19     \hyperlink{lab__functions_8h_a6f453bc035d85e967bd5032eca31a155}{input\_int}(\textcolor{stringliteral}{"Введіть кількість рядків масива (1-18665)\(\backslash\)n"}, &rows, 1, 18665);
20     \hyperlink{lab__functions_8h_a6f453bc035d85e967bd5032eca31a155}{input\_int}(\textcolor{stringliteral}{"Введіть кількість стовбців масива (1-18665)\(\backslash\)n"}, &cols, 1, 18665);
21     \textcolor{comment}{// Основний масив}
22     \textcolor{keywordtype}{int} **arr = (\textcolor{keywordtype}{int} **) malloc(rows*\textcolor{keyword}{sizeof}(\textcolor{keywordtype}{int} *));
23     \textcolor{comment}{// Суми стовбців}
24     \textcolor{keywordtype}{int} *sum\_of\_cols = (\textcolor{keywordtype}{int} *) malloc(cols*\textcolor{keyword}{sizeof}(\textcolor{keywordtype}{int}));
25     \textcolor{keywordflow}{for} (\textcolor{keywordtype}{int} i = 0; i < rows; ++i) \{
26         arr[i] = (\textcolor{keywordtype}{int} *)malloc(cols* \textcolor{keyword}{sizeof}(\textcolor{keywordtype}{int}));
27     \}
28     \textcolor{comment}{// Занулення масиву сумм колонок}
29     \textcolor{keywordflow}{for} (\textcolor{keywordtype}{int} i = 0; i < cols; ++i) \{
30         sum\_of\_cols[i] = 0;
31     \}
32     \textcolor{comment}{// Заповнення випадковими числами, підрахунок суми колонок}
33     srand(time(NULL));
34     \textcolor{keywordflow}{for} (\textcolor{keywordtype}{int} i = 0; i < rows; ++i) \{
35         \textcolor{keywordflow}{for} (\textcolor{keywordtype}{int} j = 0; j < cols; ++j) \{
36             arr[i][j] = rand()%201-100;
37             sum\_of\_cols[j]+=arr[i][j];
38         \}
39     \}
40     \textcolor{comment}{// Знаходження максимальної суми елементів}
41     \textcolor{keywordflow}{for} (\textcolor{keywordtype}{int} i = 1; i < cols; ++i) \{
42         \textcolor{keywordflow}{if} (sum\_of\_cols[max\_col]<sum\_of\_cols[i])\{
43             max\_col = i;
44         \}
45     \}
46     \textcolor{comment}{// Вивід масиву з виділеною колонкою з максимальною сумою}
47     printf(\textcolor{stringliteral}{"Масив має вигляд\(\backslash\)n"});
48     \textcolor{keywordflow}{for} (\textcolor{keywordtype}{int} i = 0; i < rows; ++i) \{
49         \textcolor{keywordflow}{for} (\textcolor{keywordtype}{int} j = 0; j < cols; ++j) \{
50             printf(\textcolor{stringliteral}{"  \(\backslash\)t"});
51             \textcolor{keywordflow}{if} (j == max\_col)\{
52                 printf(\textcolor{stringliteral}{"\(\backslash\)x1b[6;30;41m"});
53                 printf(\textcolor{stringliteral}{"%d"},arr[i][j]);
54                 printf(\textcolor{stringliteral}{"\(\backslash\)x1b[0m "});
55             \}\textcolor{keywordflow}{else}\{
56                 printf(\textcolor{stringliteral}{"%d "},arr[i][j]);
57             \}
58         \}
59         putchar(\textcolor{charliteral}{'\(\backslash\)n'});
60     \}
61     \textcolor{comment}{// Вивід масиву сум}
62     printf(\textcolor{stringliteral}{"Суми колонок\(\backslash\)n"});
63     \textcolor{keywordflow}{for} (\textcolor{keywordtype}{int} i = 0; i < cols; ++i) \{
64         printf(\textcolor{stringliteral}{"  \(\backslash\)t"});
65         \textcolor{keywordflow}{if} (i == max\_col)\{
66             printf(\textcolor{stringliteral}{"\(\backslash\)x1b[6;30;41m"});
67             printf(\textcolor{stringliteral}{"%d"},sum\_of\_cols[i]);
68             printf(\textcolor{stringliteral}{"\(\backslash\)x1b[0m "});
69         \}\textcolor{keywordflow}{else}\{
70             printf(\textcolor{stringliteral}{"%d"}, sum\_of\_cols[i]);
71         \}
72     \}
73     putchar(\textcolor{charliteral}{'\(\backslash\)n'});
74     \textcolor{comment}{// Вивід стовбця з найбільшою сумою елементів}
75     printf(\textcolor{stringliteral}{"№%d - стовбець з найбільшою сумою елементів\(\backslash\)n"}, max\_col+1);
76     \textcolor{comment}{//Вивільнення пам'яті}
77     \textcolor{keywordflow}{for} (\textcolor{keywordtype}{int} i = 0; i < rows; ++i) \{
78         free(arr[i]);
79     \}
80     free(arr);
81     free(sum\_of\_cols);
82 \}
83 
84 \textcolor{keywordtype}{int} main() \{
85     \hyperlink{lab__functions_8h_a770281b98587f9f65ca4cc75b1d061db}{to\_repeat}(\hyperlink{main_8c_a2e10594dc040249a898e2880b4c64322}{task\_1}, \textcolor{stringliteral}{"Бажаєте повторити виконання завдання?\(\backslash\)n"});
86     \textcolor{keywordflow}{return} 0;
87 \}
\end{DoxyCodeInclude}
 