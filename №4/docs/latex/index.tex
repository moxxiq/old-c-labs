\begin{center} \subsection*{Варіант №4}\end{center} 

\begin{center} \end{center}  Виконав студент групи КМ-\/82\+: {\bfseries Бубела Дмитро}~\newline
 \subsubsection*{Завдання 1.\+1 Округлення дійсного числа до значення з заданою точністю}

\paragraph*{Виконується функцією \hyperlink{main_8c_a40968dfe24ede22947096429f30444a4}{task\+\_\+1\+\_\+1()}}

\subsubsection*{Завдання 1.\+2 визначення послідовності чисел, що ділять задане число націло}

\paragraph*{Виконується функцією \hyperlink{main_8c_a97a9145d16ec992fa06404b068bd4e18}{task\+\_\+1\+\_\+2()}}

\subsubsection*{Завдання 2 Написати рекурсивну функцію, яка обчислює довжину рядка}

\paragraph*{Виконується функцією \hyperlink{main_8c_a08570886a32f0508e2a3f23c4ea06339}{task\+\_\+2()}}

\paragraph*{Нижче наведений сирцевий код прорами}


\begin{DoxyCodeInclude}
1 \textcolor{preprocessor}{#include <stdio.h>}
2 \textcolor{preprocessor}{#include <math.h>}
3 \textcolor{preprocessor}{#include <string.h>}
4 \textcolor{preprocessor}{#include <stdlib.h>}
5 \textcolor{preprocessor}{#include <values.h>}
6 \textcolor{preprocessor}{#include <ctype.h>}
7 
11 \textcolor{preprocessor}{#define clear() printf("\(\backslash\)033[H\(\backslash\)033[J")}
12 
18 \textcolor{keywordtype}{char} *\hyperlink{main_8c_a638293d509eded9d6ef7552ae1b17f2b}{message\_input}(\textcolor{keywordtype}{char} message[]);
24 \textcolor{keywordtype}{int} \hyperlink{main_8c_a9594b83ee908d195f5ff508da5c23c58}{is\_every\_digit}(\textcolor{keywordtype}{char} *str);
30 \textcolor{keywordtype}{int} \hyperlink{main_8c_a5d7849c249859dd438a37f1e6e5adf70}{is\_every\_digit\_double}(\textcolor{keywordtype}{char} *str);
37 \textcolor{keywordtype}{void} \hyperlink{main_8c_a770281b98587f9f65ca4cc75b1d061db}{to\_repeat}(\textcolor{keywordtype}{void} (*func)(\textcolor{keywordtype}{void}), \textcolor{keywordtype}{char} message[]);
43 \textcolor{keywordtype}{long} \hyperlink{main_8c_a50be0d9d5898cd1c9b3a07abb78faf4e}{pow\_10\_to}(\textcolor{keywordtype}{int} *power);
51 \textcolor{keywordtype}{void} \hyperlink{main_8c_a6f453bc035d85e967bd5032eca31a155}{input\_int}(\textcolor{keywordtype}{char} message[], \textcolor{keywordtype}{int} *number,\textcolor{keywordtype}{int} lwr\_border, \textcolor{keywordtype}{int} hir\_border);
59 \textcolor{keywordtype}{void} \hyperlink{main_8c_ac835db5eadbfefce4a51eae30806a486}{input\_long\_double}(\textcolor{keywordtype}{char} message[], \textcolor{keywordtype}{long} \textcolor{keywordtype}{double} *number,\textcolor{keywordtype}{long} \textcolor{keywordtype}{double} lwr\_border, \textcolor{keywordtype}{long} \textcolor{keywordtype}{
      double} hir\_border);
66 \textcolor{keywordtype}{long} \textcolor{keywordtype}{double} \hyperlink{main_8c_a41a04980c21ff33f1bfb435f275011db}{round\_number} (\textcolor{keywordtype}{long} \textcolor{keywordtype}{double} *num, \textcolor{keywordtype}{int} *precision);
71 \textcolor{keywordtype}{void} \hyperlink{main_8c_a2f5dd4f3afe9bea7f6dd19ed24cd9d16}{print\_dividers}(\textcolor{keywordtype}{int} *entered\_number);
78 \textcolor{keywordtype}{int} \hyperlink{main_8c_a39c41664490ca73a6f8b8224e1191711}{count\_length}(\textcolor{keywordtype}{char} phrase[],\textcolor{keywordtype}{int} i);
82 \textcolor{keywordtype}{int} \hyperlink{main_8c_a48523aba9a802fbdca6e9670a036253c}{get\_length}(\textcolor{keywordtype}{char} phrase[]);
86 \textcolor{keywordtype}{void} \hyperlink{main_8c_a2e10594dc040249a898e2880b4c64322}{task\_1}();
90 \textcolor{keywordtype}{void} \hyperlink{main_8c_a40968dfe24ede22947096429f30444a4}{task\_1\_1}();
94 \textcolor{keywordtype}{void} \hyperlink{main_8c_a97a9145d16ec992fa06404b068bd4e18}{task\_1\_2}();
98 \textcolor{keywordtype}{void} \hyperlink{main_8c_a08570886a32f0508e2a3f23c4ea06339}{task\_2}();
99 \textcolor{keywordtype}{int} main() \{
100     \hyperlink{main_8c_aff606fb64ff89d5982673319bab86b19}{clear}();
101 
102     \textcolor{keywordtype}{int} choice;
103     \textcolor{keywordflow}{do} \{
104         \hyperlink{main_8c_aff606fb64ff89d5982673319bab86b19}{clear}();
105         \hyperlink{main_8c_a6f453bc035d85e967bd5032eca31a155}{input\_int}(\textcolor{stringliteral}{"Виберіть номер завдання (1, 2):\(\backslash\)n"},
106                   &choice,
107                   1,
108                   2);
109         \textcolor{keywordflow}{switch} (choice) \{
110             \textcolor{keywordflow}{case} 1:
111                 \hyperlink{main_8c_a2e10594dc040249a898e2880b4c64322}{task\_1}();
112                 \textcolor{keywordflow}{break};
113             \textcolor{keywordflow}{case} 2:
114                 \hyperlink{main_8c_a770281b98587f9f65ca4cc75b1d061db}{to\_repeat}(\hyperlink{main_8c_a08570886a32f0508e2a3f23c4ea06339}{task\_2}, \textcolor{stringliteral}{"Бажаєте продовжити виконання завдання 2?"});
115                 \textcolor{keywordflow}{break};
116         \}
117         \hyperlink{main_8c_a6f453bc035d85e967bd5032eca31a155}{input\_int}(\textcolor{stringliteral}{"Бажаєте повторно виконати лабораторну?:\(\backslash\)n"}, &choice, 0, 1);
118     \}\textcolor{keywordflow}{while} (choice);
119     \textcolor{keywordflow}{return} 0;
120 \}
121 \textcolor{keywordtype}{void} \hyperlink{main_8c_ac835db5eadbfefce4a51eae30806a486}{input\_long\_double}(\textcolor{keywordtype}{char} message[], \textcolor{keywordtype}{long} \textcolor{keywordtype}{double} *number,\textcolor{keywordtype}{long} \textcolor{keywordtype}{double} lwr\_border, \textcolor{keywordtype}{long} \textcolor{keywordtype}{
      double} hir\_border)\{
125     \textcolor{keywordflow}{if} (lwr\_border == hir\_border)\{
126         lwr\_border = LDBL\_MIN;
127         hir\_border = LDBL\_MAX;
128     \}
129     \textcolor{keywordtype}{char} *usr\_input = \hyperlink{main_8c_a638293d509eded9d6ef7552ae1b17f2b}{message\_input}(message);
130     \textcolor{keywordtype}{int} is\_suits = sscanf(usr\_input,\textcolor{stringliteral}{"%Lf"},number);
131     \textcolor{keywordflow}{while} (!is\_suits || (!\hyperlink{main_8c_a5d7849c249859dd438a37f1e6e5adf70}{is\_every\_digit\_double}(usr\_input) && !
      \hyperlink{main_8c_a9594b83ee908d195f5ff508da5c23c58}{is\_every\_digit}(usr\_input)) || *number < lwr\_border || *number > hir\_border)\{
132         printf(\textcolor{stringliteral}{"ERROR INPUT\(\backslash\)n"});
133         usr\_input = \hyperlink{main_8c_a638293d509eded9d6ef7552ae1b17f2b}{message\_input}(message);
134         is\_suits = sscanf(usr\_input,\textcolor{stringliteral}{"%Lf"},number);
135     \}
136 \}
137 \textcolor{keywordtype}{void} \hyperlink{main_8c_a2e10594dc040249a898e2880b4c64322}{task\_1}()\{
138     \textcolor{keywordtype}{int} choice;
139     \textcolor{keywordflow}{do} \{
140         \hyperlink{main_8c_aff606fb64ff89d5982673319bab86b19}{clear}();
141         \hyperlink{main_8c_a6f453bc035d85e967bd5032eca31a155}{input\_int}(\textcolor{stringliteral}{"Виберіть номер підзавдання (1, 2):\(\backslash\)n"},
142                   &choice,
143                   1,
144                   2);
145         \textcolor{keywordflow}{switch} (choice) \{
146             \textcolor{keywordflow}{case} 1:
147                 \hyperlink{main_8c_a770281b98587f9f65ca4cc75b1d061db}{to\_repeat}(\hyperlink{main_8c_a40968dfe24ede22947096429f30444a4}{task\_1\_1},\textcolor{stringliteral}{"Бажаєте продовжити виконання підзавдання 1?"});
148                 \textcolor{keywordflow}{break};
149             \textcolor{keywordflow}{case} 2:
150                 \hyperlink{main_8c_a770281b98587f9f65ca4cc75b1d061db}{to\_repeat}(\hyperlink{main_8c_a97a9145d16ec992fa06404b068bd4e18}{task\_1\_2}, \textcolor{stringliteral}{"Бажаєте продовжити виконання підзавдання 2?"});
151                 \textcolor{keywordflow}{break};
152         \}
153         \hyperlink{main_8c_a6f453bc035d85e967bd5032eca31a155}{input\_int}(\textcolor{stringliteral}{"Бажаєте повторно виконати вибір підзавдання завдання 1?:\(\backslash\)n"}, &choice, 0, 1);
154     \}\textcolor{keywordflow}{while} (choice);
155 \}
156 \textcolor{keywordtype}{void} \hyperlink{main_8c_a40968dfe24ede22947096429f30444a4}{task\_1\_1}()\{
163     printf(\textcolor{stringliteral}{"Завдання 1.1\(\backslash\)n"});
164     \textcolor{keywordtype}{long} \textcolor{keywordtype}{double} entered\_number;
165     \textcolor{keywordtype}{int} precision;
166     \hyperlink{main_8c_ac835db5eadbfefce4a51eae30806a486}{input\_long\_double}(\textcolor{stringliteral}{"Введіть дійсне число\(\backslash\)n"}, &entered\_number, 0.0, 0.0);
167     \hyperlink{main_8c_a6f453bc035d85e967bd5032eca31a155}{input\_int}(\textcolor{stringliteral}{"Введіть бажану точність\(\backslash\)n"}, &precision, 0, INT\_MAX);
168     \textcolor{keywordtype}{long} \textcolor{keywordtype}{double} num = \hyperlink{main_8c_a41a04980c21ff33f1bfb435f275011db}{round\_number}(&entered\_number, &precision);
169     printf(\textcolor{stringliteral}{"%.*Lf\(\backslash\)n"}, precision, num);
170 \}
171 \textcolor{keywordtype}{void} \hyperlink{main_8c_a97a9145d16ec992fa06404b068bd4e18}{task\_1\_2}()\{
177     printf(\textcolor{stringliteral}{"Завдання 1.2\(\backslash\)n"});
178     \textcolor{keywordtype}{int} entered\_number;
179     \hyperlink{main_8c_a6f453bc035d85e967bd5032eca31a155}{input\_int}(\textcolor{stringliteral}{"Уведіть ціле додатнє число\(\backslash\)n"},&entered\_number,1,INT\_MAX);
180     \hyperlink{main_8c_a2f5dd4f3afe9bea7f6dd19ed24cd9d16}{print\_dividers}(&entered\_number);
181 \}
182 
183 \textcolor{keywordtype}{void} \hyperlink{main_8c_a08570886a32f0508e2a3f23c4ea06339}{task\_2}()\{
188     printf(\textcolor{stringliteral}{"Завдання 2\(\backslash\)n"});
189     \textcolor{keywordtype}{char} phrase[254];
190     printf(\textcolor{stringliteral}{"Введіть фразу (тільки латиниця (ASCII), інакше - невірний результат)\(\backslash\)n"});
191     fgets(phrase, 253, stdin);
192     printf(\textcolor{stringliteral}{"Її довжина = %d\(\backslash\)n"}, \hyperlink{main_8c_a48523aba9a802fbdca6e9670a036253c}{get\_length}(phrase)-1);
193 \}
194 
195 \textcolor{keywordtype}{char} *\hyperlink{main_8c_a638293d509eded9d6ef7552ae1b17f2b}{message\_input}(\textcolor{keywordtype}{char} message[])\{
196     \textcolor{keywordtype}{char} *usr\_input = (\textcolor{keywordtype}{char}*) malloc(\textcolor{keyword}{sizeof}(\textcolor{keywordtype}{char})*254);
197     printf(\textcolor{stringliteral}{"%s"}, message); \textcolor{comment}{// Виводимо повідомлення користувачу}
198     fgets(usr\_input, 254, stdin);
199     \textcolor{keywordflow}{return} usr\_input;
200 \}
201 
202 \textcolor{keywordtype}{int} \hyperlink{main_8c_a9594b83ee908d195f5ff508da5c23c58}{is\_every\_digit}(\textcolor{keywordtype}{char} *str)\{
207     \textcolor{keywordtype}{int} len = strlen(str);
208     \textcolor{keywordtype}{int} i = 0;
209     \textcolor{keywordflow}{if} (atoi(str)<0) i++;\textcolor{comment}{// якщо менше нуля - мінус пропускаєм}
210     \textcolor{keywordflow}{for} (i; i < len-1; ++i) \{
211         \textcolor{keywordflow}{if} (!isdigit(str[i]))\{
212             \textcolor{keywordflow}{return} 0;\textcolor{comment}{//будь-який знак - не цифра}
213         \}
214     \}
215     \textcolor{keywordflow}{return} 1;
216 \}
217 
218 \textcolor{keywordtype}{int} \hyperlink{main_8c_a5d7849c249859dd438a37f1e6e5adf70}{is\_every\_digit\_double}(\textcolor{keywordtype}{char} *str)\{
223     \textcolor{keywordtype}{int} len = strlen(str);
224     \textcolor{keywordtype}{long} \textcolor{keywordtype}{double} number;
225     sscanf(str,\textcolor{stringliteral}{"%Lf"},&number);
226     \textcolor{keywordtype}{int} i = 0, dot\_counter = 0;
227     \textcolor{keywordflow}{if} (number<0.0) i++;\textcolor{comment}{// якщо менше нуля - мінус пропускаєм}
228     \textcolor{keywordflow}{for} (i; i < len-1; ++i) \{
229         \textcolor{keywordflow}{if} (!isdigit(str[i]))\{
230             \textcolor{keywordflow}{if} (str[i]==\textcolor{charliteral}{'.'}) \{
231                 dot\_counter++;
232             \}\textcolor{keywordflow}{else}\{
233                 \textcolor{keywordflow}{return} 0;\textcolor{comment}{//будь-який знак - не цифра або крапка (максимум 1)}
234             \}
235         \}
236     \}
237     \textcolor{keywordflow}{if} (dot\_counter-1) \textcolor{keywordflow}{return} 0;
238     \textcolor{keywordflow}{return} 1;
239 \}
240 
241 \textcolor{keywordtype}{void} \hyperlink{main_8c_a6f453bc035d85e967bd5032eca31a155}{input\_int}(\textcolor{keywordtype}{char} message[], \textcolor{keywordtype}{int} *number,\textcolor{keywordtype}{int} lwr\_border, \textcolor{keywordtype}{int} hir\_border)\{
245     \textcolor{keywordflow}{if} (lwr\_border == hir\_border)\{
246         lwr\_border = INT\_MIN;
247         hir\_border = INT\_MAX;
248     \}
249     \textcolor{keywordtype}{char} *usr\_input = \hyperlink{main_8c_a638293d509eded9d6ef7552ae1b17f2b}{message\_input}(message);
250     \textcolor{keywordtype}{int} is\_suits = sscanf(usr\_input,\textcolor{stringliteral}{"%d"},number);
251     \textcolor{keywordflow}{while} (!is\_suits || !\hyperlink{main_8c_a9594b83ee908d195f5ff508da5c23c58}{is\_every\_digit}(usr\_input) || *number < lwr\_border || *number > 
      hir\_border)\{
252         printf(\textcolor{stringliteral}{"ERROR INPUT\(\backslash\)n"});
253         usr\_input = \hyperlink{main_8c_a638293d509eded9d6ef7552ae1b17f2b}{message\_input}(message);
254         is\_suits = sscanf(usr\_input,\textcolor{stringliteral}{"%d"},number);
255     \}
256 \}
257 \textcolor{keywordtype}{void} \hyperlink{main_8c_a770281b98587f9f65ca4cc75b1d061db}{to\_repeat}(\textcolor{keywordtype}{void} (*func)(\textcolor{keywordtype}{void}), \textcolor{keywordtype}{char} message[])\{
262     \textcolor{keywordtype}{int} choice;
263     \textcolor{keywordflow}{do}\{
264         (*func)();
265         \hyperlink{main_8c_a6f453bc035d85e967bd5032eca31a155}{input\_int}(message,&choice,0,1);
266     \}\textcolor{keywordflow}{while} (choice);
267 \}
268 \textcolor{keywordtype}{int} \hyperlink{main_8c_a48523aba9a802fbdca6e9670a036253c}{get\_length}(\textcolor{keywordtype}{char} phrase[])\{
274     \textcolor{keywordflow}{return} \hyperlink{main_8c_a39c41664490ca73a6f8b8224e1191711}{count\_length}(phrase,0);
275 \}
276 
277 \textcolor{keywordtype}{long} \hyperlink{main_8c_a50be0d9d5898cd1c9b3a07abb78faf4e}{pow\_10\_to}(\textcolor{keywordtype}{int} *power)\{
278 
279     \textcolor{keywordtype}{long} ten = 1;
280     \textcolor{keywordflow}{for} (\textcolor{keywordtype}{int} i = 0; i < *power; ++i) \{
281         ten *= 10;
282     \}
283     \textcolor{keywordflow}{return} ten;
284 \}
285 \textcolor{keywordtype}{int} \hyperlink{main_8c_a39c41664490ca73a6f8b8224e1191711}{count\_length}(\textcolor{keywordtype}{char} phrase[],\textcolor{keywordtype}{int} i)\{
289     \textcolor{keywordflow}{if} (!phrase[i])\{
290         \textcolor{keywordflow}{return} 0;
291     \}
292     \textcolor{keywordtype}{int} sum = 1 + \hyperlink{main_8c_a39c41664490ca73a6f8b8224e1191711}{count\_length}(phrase,i+1);
293     \textcolor{keywordflow}{return} sum;
294 \}
295 
296 \textcolor{keywordtype}{void} \hyperlink{main_8c_a2f5dd4f3afe9bea7f6dd19ed24cd9d16}{print\_dividers}(\textcolor{keywordtype}{int} *entered\_number)\{
301     printf(\textcolor{stringliteral}{"Дільники числа:\(\backslash\)n1\(\backslash\)n"});
302     \textcolor{keywordflow}{for} (\textcolor{keywordtype}{int} i = 2; i < *entered\_number/2; ++i) \{
303         \textcolor{keywordflow}{if} (!(*entered\_number%i))\{
304             printf(\textcolor{stringliteral}{"%d\(\backslash\)n"}, i);
305         \}
306     \}
307     \textcolor{keywordflow}{if} (*entered\_number>1) printf(\textcolor{stringliteral}{"%d\(\backslash\)n"},*entered\_number);
308 \}
309 
310 \textcolor{keywordtype}{long} \textcolor{keywordtype}{double} \hyperlink{main_8c_a41a04980c21ff33f1bfb435f275011db}{round\_number} (\textcolor{keywordtype}{long} \textcolor{keywordtype}{double} *num, \textcolor{keywordtype}{int} *precision)\{
317     \textcolor{comment}{// створено додаткові змінні для пояснення суті та полегшення відладки}
318     \textcolor{keywordtype}{long} accuracy = \hyperlink{main_8c_a50be0d9d5898cd1c9b3a07abb78faf4e}{pow\_10\_to}(precision); \textcolor{comment}{// значення точності перетворене в десятки}
319     \textcolor{keywordtype}{long} \textcolor{keywordtype}{double} math\_round\_addition = 0.5*(1/(\textcolor{keywordtype}{long} double)accuracy);\textcolor{comment}{// використовуєтся заокруглення за
       математичними правилами (= [x+0.5])}
320     \textcolor{keywordtype}{long} \textcolor{keywordtype}{double} answer = (long)(((*num)+math\_round\_addition)*accuracy);\textcolor{comment}{// відкидаються решта "нулів"}
321     answer /= (\textcolor{keywordtype}{long} double)accuracy ;\textcolor{comment}{// власне поділ на цілу і дробову частини}
322     \textcolor{keywordflow}{return} answer;
323 \}
\end{DoxyCodeInclude}
 