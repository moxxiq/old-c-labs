\begin{center} \subsection*{Варіант №4}\end{center} 

\begin{center} \end{center}  Виконав студент групи КМ-\/82\+: {\bfseries Бубела Дмитро}~\newline
 На початку виконяння програми виводиться інформація пов\textquotesingle{}язана з завданням(и) та автором.~\newline
Викликається функція \hyperlink{main_8c_a967daaa1c8b4c7d21c4ccf7a81fbe67d}{choose\+\_\+lab()} \subsubsection*{Завдання 1.\+1. Обчислити висоту паралелепіпеда.}

\paragraph*{Виконується функцією \hyperlink{main_8c_ad0ae7d23ef39095cda907019fd94ed96}{task1sub1()}}

Спочатку ініціалізуєм потрібні нам змінні.~\newline
Вводимо об\textquotesingle{}єм та площу ({\ttfamily do} {\ttfamily while} цикли використовуєм для перевірки на невід\textquotesingle{}ємність).~\newline
Здійснюємо обчислення. Виводим висоту паралелограма 
\begin{DoxyCode}
printf(\textcolor{stringliteral}{"Висота паралелограма (H = V/S) = "});
printf(\textcolor{stringliteral}{"%.3lf\(\backslash\)n"}, H);
\end{DoxyCode}
 Реалізовано виконання функції ще раз за бажанням користувача. \subsubsection*{Завдання 1.\+2. Обчислити Евклідову відстань між двома точками.}

\paragraph*{Виконується функцією \hyperlink{main_8c_ab6b6fe73040966990d9129dbf55cb110}{task1sub2()}}

Ініціалізуєм змінні, вводим відповідні координати за допомогою \hyperlink{main_8c_aa9f623385e5c1c8ac44a985d44cf3c5a}{input\+D()}~\newline
Обчислюєм відстань між точками $(x_1,y_1)$ і $(x_2,y_2) = \sqrt{(x_2-x_1)^2+(y_2-y_1)^2}$ \subsubsection*{Завдання 2}

Завдання складається з двох задач. У рамках завдання потрібно скласти дві підпрограми\+:
\begin{DoxyItemize}
\item перша підпрограма уводить складові частини структури даних, наdеденої у відповідному варіанті індивідуального завдання, і формує з них задану упаковану структуру;
\item друга підпрограма уводить упаковану структуру як шістнадцяткове число й виводить значення окремих її складових частин.
\end{DoxyItemize}

Формат команди завантаження/збереження в обчислювальній системі має вигляд\+: \tabulinesep=1mm
\begin{longtabu} spread 0pt [c]{*{17}{|X[-1]}|}
\hline
\rowcolor{\tableheadbgcolor}\textbf{ № розрядів }&\textbf{ 15 }&\textbf{ 14 }&\textbf{ 13 }&\textbf{ 12 }&\textbf{ 11 }&\textbf{ 10 }&\textbf{ 9 }&\textbf{ 8 }&\textbf{ 7 }&\textbf{ 6 }&\textbf{ 5 }&\textbf{ 4 }&\textbf{ 3 }&\textbf{ 2 }&\textbf{ 1 }&\textbf{ 0  }\\\cline{1-17}
\endfirsthead
\hline
\endfoot
\hline
\rowcolor{\tableheadbgcolor}\textbf{ № розрядів }&\textbf{ 15 }&\textbf{ 14 }&\textbf{ 13 }&\textbf{ 12 }&\textbf{ 11 }&\textbf{ 10 }&\textbf{ 9 }&\textbf{ 8 }&\textbf{ 7 }&\textbf{ 6 }&\textbf{ 5 }&\textbf{ 4 }&\textbf{ 3 }&\textbf{ 2 }&\textbf{ 1 }&\textbf{ 0  }\\\cline{1-17}
\endhead
значення &1 &1 &1 &0 &0 &0 &1 &D &R &R &R &R &A &A &A &A \\\cline{1-17}
\end{longtabu}
Підпрограма 1 \hyperlink{main_8c_a49562a6161d394ce5f96b6731e07e440}{task2sub1()} Підпрограма2 \hyperlink{main_8c_a08a44a4f43367b221db5d4e7b2657623}{task2sub2()}~\newline
\subsubsection*{Завдання 2.\+1.}

Ініціалізація змінних.~\newline
Для зчитування вводу користувача використовується допоміжна змінна {\ttfamily temp} На кожному етапі введення змінна є корректною, використовується функція \hyperlink{main_8c_aa9f623385e5c1c8ac44a985d44cf3c5a}{input\+D()}~\newline
За допомогою бітових операцій до змінної {\ttfamily Unit\+State\+Word} записуються команди~\newline
Команда виводиться.~\newline
Реалізовано виконання функції ще раз за бажанням користувача. \subsubsection*{Завдання 2.\+2.}

Вводиться шістнадцядкове число -\/ код стану та перевіяється на відповідність заданій у варіанті команді.~\newline
Якщо число не є командою, яку ми хочем розібрати -\/ виводим користувачу повідомлення. При введенні вірної команди -\/ вона розбирається на операнди та виводиться користувачу. Реалізовано виконання функції ще раз за бажанням користувача. \paragraph*{Нижче наведений сирцевий код прорами}


\begin{DoxyCodeInclude}
1 \textcolor{preprocessor}{#include <stdio.h>}
2 \textcolor{preprocessor}{#include <math.h>}
3 \textcolor{preprocessor}{#include <stdlib.h>}
4 
21 \textcolor{keywordtype}{void} \hyperlink{main_8c_a967daaa1c8b4c7d21c4ccf7a81fbe67d}{choose\_lab}();
26 \textcolor{keywordtype}{void} \hyperlink{main_8c_afde07648040c326129670547738a0c86}{task1}();
38 \textcolor{keywordtype}{void} \hyperlink{main_8c_ad0ae7d23ef39095cda907019fd94ed96}{task1sub1}();
44 \textcolor{keywordtype}{void} \hyperlink{main_8c_ab6b6fe73040966990d9129dbf55cb110}{task1sub2}();
55 \textcolor{keywordtype}{void} \hyperlink{main_8c_afb35a54f26606b4808ac0a8d9ad55433}{task2}();
65 \textcolor{keywordtype}{void} \hyperlink{main_8c_a49562a6161d394ce5f96b6731e07e440}{task2sub1}();
74 \textcolor{keywordtype}{void} \hyperlink{main_8c_a08a44a4f43367b221db5d4e7b2657623}{task2sub2}();
79 \textcolor{keywordtype}{char} \hyperlink{main_8c_a20531ce01d5668d3c0eafb037eb3c514}{get1char}();
89 \textcolor{keywordtype}{void} \hyperlink{main_8c_aa9f623385e5c1c8ac44a985d44cf3c5a}{inputD}(\textcolor{keywordtype}{char} message[], \textcolor{keywordtype}{int} *number);
95 \textcolor{keywordtype}{void} \hyperlink{main_8c_a67a83febb2bdcd536c9edefe419afb2e}{inputF}(\textcolor{keywordtype}{char} message[],\textcolor{keywordtype}{double} *number);
96 
97 \textcolor{keywordtype}{int} main() \{
98     system(\textcolor{stringliteral}{"clear"});
99     printf(\textcolor{stringliteral}{"ЛАБОРАТОРНА РОБОТА №1\(\backslash\)n"}
100            \textcolor{stringliteral}{"БАЗОВІ ТИПИ ДАНИХ, УВЕДЕННЯ-ВИВЕДЕННЯ,\(\backslash\)n"}
101            \textcolor{stringliteral}{"БІТОВІ ОПЕРАЦІЇ, ОПЕРАЦІЇ ЗСУВУ\(\backslash\)n"}
102            \textcolor{stringliteral}{"Виконав студент групи КМ-82\(\backslash\)n"}
103            \textcolor{stringliteral}{"Бубела Дмитро\(\backslash\)n"}
104            \textcolor{stringliteral}{"Варіант №4\(\backslash\)n"});
105     \hyperlink{main_8c_a967daaa1c8b4c7d21c4ccf7a81fbe67d}{choose\_lab}();
106     \textcolor{keywordflow}{return} 0;
107 \}
108 
109 \textcolor{keywordtype}{void} \hyperlink{main_8c_aa9f623385e5c1c8ac44a985d44cf3c5a}{inputD}(\textcolor{keywordtype}{char} message[], \textcolor{keywordtype}{int} *number)
110 \{
111     \textcolor{keywordtype}{int} isD = 0;
112     *number = 0; \textcolor{comment}{// Зануляєм}
113     \textcolor{keywordflow}{do}
114     \{
115         printf(\textcolor{stringliteral}{"%s"}, message); \textcolor{comment}{// Виводимо повідомлення користувачу}
116         isD = scanf(\textcolor{stringliteral}{"%d"}, number);\textcolor{comment}{// якщо ввід коректний isD = 1}
117         \textcolor{keywordflow}{while} (getchar()!=\textcolor{charliteral}{'\(\backslash\)n'}) \textcolor{keywordflow}{continue}; \textcolor{comment}{// дуже важлива функція для очищення вхідного потоку}
118     \}
119     \textcolor{keywordflow}{while} (!isD);\textcolor{comment}{// поки корисувач не введе значення - повторюєм виконання циклу}
120 \}
121 
122 \textcolor{keywordtype}{void} \hyperlink{main_8c_a67a83febb2bdcd536c9edefe419afb2e}{inputF}(\textcolor{keywordtype}{char} message[],\textcolor{keywordtype}{double} *number)
123 \{
124     \textcolor{keywordtype}{int} isF = 0;
125     *number = 0.0;
126     \textcolor{keywordflow}{do}
127     \{
128         printf(\textcolor{stringliteral}{"%s"}, message);
129         isF = scanf(\textcolor{stringliteral}{"%lf"}, number);
130         \textcolor{keywordflow}{while} (getchar()!=\textcolor{charliteral}{'\(\backslash\)n'}) \textcolor{keywordflow}{continue};
131     \}
132     \textcolor{keywordflow}{while} (!isF);
133 \}
134 
135 \textcolor{keywordtype}{char} \hyperlink{main_8c_a20531ce01d5668d3c0eafb037eb3c514}{get1char}()
136 \{
137     \textcolor{keywordtype}{char} i = getchar();
138     \textcolor{keywordflow}{while} (getchar()!=\textcolor{charliteral}{'\(\backslash\)n'}) \textcolor{keywordflow}{continue};
139     \textcolor{keywordflow}{return} i;
140 \}
141 
142 \textcolor{keywordtype}{void} \hyperlink{main_8c_a967daaa1c8b4c7d21c4ccf7a81fbe67d}{choose\_lab}()
143 \{
144     \textcolor{keywordtype}{int} labNumber;
145     \textcolor{keywordflow}{do}
146     \{
147         \hyperlink{main_8c_aa9f623385e5c1c8ac44a985d44cf3c5a}{inputD}(\textcolor{stringliteral}{"Виберіть завдання з лабораторної роботи 1, 2:\(\backslash\)n"}, &labNumber);
148     \}
149     \textcolor{keywordflow}{while} (labNumber < 1 || labNumber > 2);
150     \textcolor{comment}{//Вимагаєм у користувача цифру 1 або 2}
151     \textcolor{keywordflow}{switch} (labNumber)
152     \{
153         \textcolor{comment}{//Виконуєм завдання лабораторнної}
154         \textcolor{keywordflow}{case} 1: \hyperlink{main_8c_afde07648040c326129670547738a0c86}{task1}(); \textcolor{keywordflow}{break};
155         \textcolor{keywordflow}{case} 2: \hyperlink{main_8c_afb35a54f26606b4808ac0a8d9ad55433}{task2}(); \textcolor{keywordflow}{break};
156     \}
157     printf(\textcolor{stringliteral}{"Хочете продовжити виконання лабораторної роботи? (0 / 1)\(\backslash\)n"});
158     \textcolor{keywordflow}{if} (\hyperlink{main_8c_a20531ce01d5668d3c0eafb037eb3c514}{get1char}() == \textcolor{charliteral}{'1'})
159     \{
160         \hyperlink{main_8c_a967daaa1c8b4c7d21c4ccf7a81fbe67d}{choose\_lab}();
161     \}
162 \}
163 
164 \textcolor{keywordtype}{void} \hyperlink{main_8c_afde07648040c326129670547738a0c86}{task1}()
165 \{
166     \textcolor{keywordtype}{int} subTask = 0;
167     \textcolor{keywordflow}{do}
168     \{
169         \hyperlink{main_8c_aa9f623385e5c1c8ac44a985d44cf3c5a}{inputD}(\textcolor{stringliteral}{"Введіть номер підзавдання: 1, 2\(\backslash\)n"}, &subTask);
170     \}
171     \textcolor{keywordflow}{while} (subTask != 1 && subTask != 2);
172     \textcolor{keywordflow}{switch} (subTask)
173     \{
174         \textcolor{keywordflow}{case} 1: \hyperlink{main_8c_ad0ae7d23ef39095cda907019fd94ed96}{task1sub1}(); \textcolor{keywordflow}{break};
175         \textcolor{keywordflow}{case} 2: \hyperlink{main_8c_ab6b6fe73040966990d9129dbf55cb110}{task1sub2}(); \textcolor{keywordflow}{break};
176     \}
177     printf(\textcolor{stringliteral}{"Бажаєте повторити вибір першого завдання? (0/1)\(\backslash\)n"});
178     \textcolor{keywordflow}{if} (\hyperlink{main_8c_a20531ce01d5668d3c0eafb037eb3c514}{get1char}() == \textcolor{charliteral}{'1'})
179     \{
180         \hyperlink{main_8c_afde07648040c326129670547738a0c86}{task1}();
181     \}
182 \}
183 
184 \textcolor{keywordtype}{void} \hyperlink{main_8c_ad0ae7d23ef39095cda907019fd94ed96}{task1sub1}()
185 \{
186     \textcolor{keywordtype}{double} V,S,H;\textcolor{comment}{//Відповідні дані, що необхідні для здійснення обчислень}
187     printf(\textcolor{stringliteral}{"Я програма для обчислення висоти паралелепіпеда за формулою H = V/S\(\backslash\)n"});
188     \textcolor{keywordflow}{do}
189     \{
190         \hyperlink{main_8c_a67a83febb2bdcd536c9edefe419afb2e}{inputF}(\textcolor{stringliteral}{"Введіть об'єм паралелограма:"}, &V);
191     \}
192     \textcolor{keywordflow}{while} (V<0);
193     \textcolor{keywordflow}{do}
194     \{
195         \hyperlink{main_8c_a67a83febb2bdcd536c9edefe419afb2e}{inputF}(\textcolor{stringliteral}{"Введіть площу основи паралелограма:"}, &S);
196     \}
197     \textcolor{keywordflow}{while} (S<0);
198     H = V/S;
199     printf(\textcolor{stringliteral}{"Висота паралелограма (H = V/S) = "});
200     printf(\textcolor{stringliteral}{"%.3lf\(\backslash\)n"}, H);
201     printf(\textcolor{stringliteral}{"Бажаєте повторити це завдання? (0/1)\(\backslash\)n"});
202     \textcolor{keywordflow}{if} (\hyperlink{main_8c_a20531ce01d5668d3c0eafb037eb3c514}{get1char}() == \textcolor{charliteral}{'1'})
203     \{
204         \hyperlink{main_8c_ad0ae7d23ef39095cda907019fd94ed96}{task1sub1}();
205     \}
206 \}
207 
208 \textcolor{keywordtype}{void} \hyperlink{main_8c_ab6b6fe73040966990d9129dbf55cb110}{task1sub2}()
209 \{
210     \textcolor{keywordtype}{double} distance;
211     \textcolor{keywordtype}{int} x1,x2,y1,y2;
212     printf(\textcolor{stringliteral}{"Я програма для обчислення відстаней між точками.\(\backslash\)n"});
213     printf(\textcolor{stringliteral}{"Введіть координати 2х точок (x1;y1),(x1;y2)\(\backslash\)n"});
214     \hyperlink{main_8c_aa9f623385e5c1c8ac44a985d44cf3c5a}{inputD}(\textcolor{stringliteral}{"x1 = "},&x1);
215     \hyperlink{main_8c_aa9f623385e5c1c8ac44a985d44cf3c5a}{inputD}(\textcolor{stringliteral}{"y1 = "},&y1);
216     \hyperlink{main_8c_aa9f623385e5c1c8ac44a985d44cf3c5a}{inputD}(\textcolor{stringliteral}{"x2 = "},&x2);
217     \hyperlink{main_8c_aa9f623385e5c1c8ac44a985d44cf3c5a}{inputD}(\textcolor{stringliteral}{"y2 = "},&y2);
218     distance = sqrt((x2 - x1)*(x2 - x1)+(y2 - y1)*(y2 - y1));
219     printf(\textcolor{stringliteral}{"Відстань між точками = %.3lf\(\backslash\)n"}, distance);
220     printf(\textcolor{stringliteral}{"Бажаєте повторити це завдання? (0/1)\(\backslash\)n"});
221     \textcolor{keywordflow}{if} (\hyperlink{main_8c_a20531ce01d5668d3c0eafb037eb3c514}{get1char}() == \textcolor{charliteral}{'1'})
222     \{
223         \hyperlink{main_8c_ab6b6fe73040966990d9129dbf55cb110}{task1sub2}();
224     \}
225 \}
226 
227 \textcolor{keywordtype}{void} \hyperlink{main_8c_afb35a54f26606b4808ac0a8d9ad55433}{task2}()
228 \{
229     \textcolor{keywordtype}{int} subTask = 0;
230     \textcolor{keywordflow}{do}
231     \{
232         \hyperlink{main_8c_aa9f623385e5c1c8ac44a985d44cf3c5a}{inputD}(\textcolor{stringliteral}{"Введіть номер підзавдання (завантаження/збереження): 1, 2\(\backslash\)n"}, &subTask);
233     \}
234     \textcolor{keywordflow}{while} (subTask != 1 && subTask != 2);
235     \textcolor{keywordflow}{switch} (subTask)
236     \{
237         \textcolor{keywordflow}{case} 1: \hyperlink{main_8c_a49562a6161d394ce5f96b6731e07e440}{task2sub1}(); \textcolor{keywordflow}{break};
238         \textcolor{keywordflow}{case} 2: \hyperlink{main_8c_a08a44a4f43367b221db5d4e7b2657623}{task2sub2}(); \textcolor{keywordflow}{break};
239     \}
240     printf(\textcolor{stringliteral}{"Бажаєте повторити вибір першого завдання? (0/1)\(\backslash\)n"});
241     \textcolor{keywordflow}{if} (\hyperlink{main_8c_a20531ce01d5668d3c0eafb037eb3c514}{get1char}() == \textcolor{charliteral}{'1'})
242     \{
243         \hyperlink{main_8c_afb35a54f26606b4808ac0a8d9ad55433}{task2}();
244     \}
245 \}
246 
247 \textcolor{keywordtype}{void} \hyperlink{main_8c_a49562a6161d394ce5f96b6731e07e440}{task2sub1}()
248 \{
249     \textcolor{keywordtype}{char} D,R,A; \textcolor{comment}{// напрям у регістр/пам'ять // перший операнд (регістр) // регістр адреси другого операнда}
250     \textcolor{keywordtype}{unsigned} \textcolor{keywordtype}{int} UnitStateWord; \textcolor{comment}{//слово стану}
251     \textcolor{keywordtype}{int} temp;
252     printf(\textcolor{stringliteral}{"Я програма, що формує команду завантаження/збереження в обчислювальній системі\(\backslash\)n"});
253     \textcolor{keywordflow}{do}
254     \{
255         \hyperlink{main_8c_aa9f623385e5c1c8ac44a985d44cf3c5a}{inputD}(\textcolor{stringliteral}{"Уведіть напрям запису у регістр/пам'ять (1/0)\(\backslash\)n"}, &temp);
256     \}
257     \textcolor{keywordflow}{while} (temp != 1 && temp != 0);
258     D = temp;
259     \textcolor{keywordflow}{do}
260     \{
261         \hyperlink{main_8c_aa9f623385e5c1c8ac44a985d44cf3c5a}{inputD}(\textcolor{stringliteral}{"Уведіть перший операнд (регістр) (0-15)\(\backslash\)n"}, &temp);
262     \}
263     \textcolor{keywordflow}{while} (temp < 0 || temp > 15);
264     R = temp;
265     \textcolor{keywordflow}{do}
266     \{
267         \hyperlink{main_8c_aa9f623385e5c1c8ac44a985d44cf3c5a}{inputD}(\textcolor{stringliteral}{"Уведіть регістр адреси другого операнда (0-15)\(\backslash\)n"}, &temp);
268     \}
269     \textcolor{keywordflow}{while} (temp < 0 || temp > 15);
270     A = temp;
271     UnitStateWord = 0x71<<9;
272     UnitStateWord |= (D & 1)<<8;
273     UnitStateWord |= (R & 0xF)<<4;
274     UnitStateWord |= (A & 0xF);
275     printf(\textcolor{stringliteral}{"Слово стану пристрою %04x\(\backslash\)n"},UnitStateWord);
276     printf(\textcolor{stringliteral}{"Бажаєте повторити це завдання? (0/1)\(\backslash\)n"});
277     \textcolor{keywordflow}{if} (\hyperlink{main_8c_a20531ce01d5668d3c0eafb037eb3c514}{get1char}() == \textcolor{charliteral}{'1'})
278     \{
279         \hyperlink{main_8c_a49562a6161d394ce5f96b6731e07e440}{task2sub1}();
280     \}
281 \}
282 
283 \textcolor{keywordtype}{void} \hyperlink{main_8c_a08a44a4f43367b221db5d4e7b2657623}{task2sub2}()
284 \{
285     \textcolor{keywordtype}{unsigned} \textcolor{keywordtype}{int} UnitStateWord; \textcolor{comment}{//слово стану}
286     \textcolor{keywordtype}{int} isX;
287     printf(\textcolor{stringliteral}{"Я програма, що розбирає команду завантаження/збереження в обчислювальній системі\(\backslash\)n"});
288     \textcolor{keywordflow}{do}
289     \{
290         printf(\textcolor{stringliteral}{"Уведіть cлово стану пристрою\(\backslash\)n"}
291                \textcolor{stringliteral}{"(16-кове число від 0 до e3ff):\(\backslash\)n"});
292         isX = scanf(\textcolor{stringliteral}{"%x"}, &UnitStateWord);
293         \textcolor{keywordflow}{while} (getchar()!=\textcolor{charliteral}{'\(\backslash\)n'}) \textcolor{keywordflow}{continue};
294     \}
295     \textcolor{keywordflow}{while} (!isX || UnitStateWord>0xFFFF);
296     \textcolor{keywordflow}{if} (UnitStateWord>>8 != 0xe3 && UnitStateWord>>8 != 0xe2)
297     \{
298         printf(\textcolor{stringliteral}{"Це не команда збереження/завантаження!!!\(\backslash\)n"});
299     \}
300     \textcolor{keywordflow}{else}
301     \{
302         printf(\textcolor{stringliteral}{"Напрямок передачі = %d\(\backslash\)n"}, (UnitStateWord>>8) & 1);
303         printf(\textcolor{stringliteral}{"Перший операнд (регістр) = %d\(\backslash\)n"}, (UnitStateWord>>4) & 0xF);
304         printf(\textcolor{stringliteral}{"Регістр адреси другого операнда = %d\(\backslash\)n"}, UnitStateWord & 0xF);
305     \}
306     printf(\textcolor{stringliteral}{"Бажаєте повторити це завдання? (0/1)\(\backslash\)n"});
307     \textcolor{keywordflow}{if} (\hyperlink{main_8c_a20531ce01d5668d3c0eafb037eb3c514}{get1char}() == \textcolor{charliteral}{'1'})
308     \{
309         \hyperlink{main_8c_a08a44a4f43367b221db5d4e7b2657623}{task2sub2}();
310     \}
311 \}
\end{DoxyCodeInclude}
 